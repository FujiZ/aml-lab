\documentclass[a4paper,UTF8]{article}
\usepackage{ctex}
\usepackage[margin=1.25in]{geometry}
\usepackage{color}
\usepackage{graphicx}
\usepackage{amssymb}
\usepackage{amsmath}
\usepackage{amsthm}
%\usepackage[thmmarks, amsmath, thref]{ntheorem}
\theoremstyle{definition}
\newtheorem*{solution}{Solution}
\newtheorem*{prove}{Proof}
\usepackage{multirow}
\usepackage{url}
\usepackage{enumerate}
\usepackage{algorithm}
\usepackage{algorithmic}
\renewcommand{\algorithmicrequire}{\textbf{Input:}}
\renewcommand{\algorithmicensure}{\textbf{Procedure:}}
\bibliographystyle{plain}
\renewcommand\refname{参考文献}
\DeclareMathOperator*{\argmax}{argmax} % thin space, limits underneath in displays

%--

%--
\begin{document}
\title{实验2. 隐马尔科夫模型实践}
\author{MG1733098,周华平,\url{zhp@smail.nju.edu.cn}}
\maketitle

\section*{综述}

隐马尔科夫模型(hidden Markov model, HMM)是可用于标注问题的统计学习模型,
描述由隐藏的马尔科夫链随机生成观测序列的过程,属于生成模型。

本实验首先针对一个已经训练好的HMM,实现维特比算法,通过动态规划的思想对模型进⾏推断;
其次,针对参数未知的HMM,通过Baum–Welch算法对数据进行学习与训练,
其中我们需要实现HMM的前向和后向算法。

\section*{实验一.}

维特比算法实际是用动态规划解隐马尔科夫模型预测问题,
即用动态规划求概率最大路径(最优路径)。
这时一条路径对应着一个状态序列。

根据动态规划原理,最优路径具有这样的特性:
如果最优路径在时刻$t$通过节点$i_t^*$,
那么这一路径从节点$i_t^*$到终点$i_T^*$的部分路径,
对于从$i_t^*$到$i_T^*$的所有可能的部分路径来说,必须是最优的。
因为假如不是这样,那么从$i_t^*$到$i_T^*$就有另一条更好的部分路径存在,
如果把它和从$i_t^*$到达$i_T^*$的部分路径连接起来,
就会形成一条比原来的路径更优的路径,这是矛盾的。
依据这一原理,我们只需从时刻$t=1$开始,
递推地计算在时刻$t$状态为$i$的各条部分路径的最大概率,
直至得到时刻$t=T$状态为$i$的各条路径的最大概率。
时刻$t=T$的最大概率即为最优路径的概率$P^*$,
最优路径的终结点$i_T^*$也同时得到。
之后,为了找出最优路径的各个节点,从终结点$i_T^*$开始,
由后向前逐步求得节点$i_{T-1}^*,\dots,i_1^*$,
得到最优路径$I^*=(i_1^*,i_2^*,\dots,i_T^*)$。
这就是维特比算法。

首先导入两个变量$\delta$和$\psi$。
定义在时刻$t$状态为$i$的所有单个路径$(i_1,i_2,\dots,i_t)$中概率最大值为
\[
	\delta_t(i)=\max_{i_1,i_2,\dots,i_{t-1}} P(i_t=i,i_{t-1},\dots,i_1,o_t,\dots,o_1 \vert \lambda),\qquad
	i=1,2,\dots,N
\]

由定义可得变量$\delta$的递推公式:
\begin{align*}
	\delta_{t+1}(i)&=\max_{i_1,i_2,\dots,i_t} P(i_{t+1}=i,i_t,\dots,i_1,o_{t+1},\dots,o_1 \vert \lambda) \\
	&=\max_{1 \leq j \leq N}[\delta_t(j)a_{ji}]b_i(o_{t+1}),\qquad
	i=1,2,\dots,N; t=1,2,\dots,T-1
\end{align*}

定义在时刻$t$状态为$i$的所有单个路径$(i_1,i_2,\dots,i_{t-1},i)$
中概率最大的路径的第$t-1$个节点为
\[
	\psi_t(i)=\argmax_{1 \leq j \leq N}[\delta_{t-1}(j)a_{ji}],\qquad
	i=1,2,\dots,N
\]

\subsection*{维特比算法}

输入:模型$\lambda=(A,B,\pi)$和观测$O=(o_1,o_2,\dots,o_T)$

输出:最优路径$I^*=(i_1^*,i_2^*,\dots,i_T^*)$

(1)初始化
\begin{gather*}
	\delta_1(i)=\pi_ib_i(o_1),\qquad i=1,2,\dots,N \\
	\psi_1(i)=0,\qquad i=1,2,\dots,N
\end{gather*}

(2)递推。对$t=2,3,\dots,T$
\begin{gather*}
	\delta_t(i)=\max_{1 \leq j \leq N}[\delta_{t-1}(j)a_{ji}]b_i(o_i),
	\qquad i=1,2,\dots,N \\
	\psi_t(i)=\argmax_{1 \leq j \leq N}[\delta_{t-1}(j)a_{ji}],
	\qquad i=1,2,\dots,N
\end{gather*}

(3)终止
\begin{gather*}
	P^*=\max_{1 \leq i \leq N}\delta_T(i) \\
	i_T^*=\argmax_{1 \leq i \leq N}[\delta_T(i)]
\end{gather*}

(4)最优路径回溯。对$t=T-1,T-2,\dots,1$
\[
	i_t^*=\psi_{t+1}(i_{t+1}^*)
\]
求得最优路径$I^*=(i_1^*,i_2^*,\dots,i_T^*)$。

\section*{实验二.}

给定隐马尔科夫模型$\lambda$,定义到时刻$t$部分观测序列为$o_1,o_2,\dots,o_t$
且状态为$q_i$的概率为前向概率,记作
\[
	\alpha_t(i)=P(o_1,o_2,\dots,o_t,i_t=q_i \vert \lambda)
\]

可以递推地求得前向概率$\alpha_t(i)$及观测序列概率$P(O \vert \lambda)$。

\subsection*{观测序列概率的前向算法}

输入:隐马尔科夫模型$\lambda$,观测序列$O$

输出:观测序列概率$P(O \vert \lambda)$

(1)初值
\[
	\alpha_1(i)=\pi_ib_i(o_1),\qquad i=1,2,\dots,N
\]

(2)递推。对$t=1,2,\dots,T-1$
\[
	\alpha_{t+1}(i)=[\sum_{j=1}^N \alpha_t(j)a_{ji}]b_i(o_{t+1}),
	\qquad i=1,2,\dots,N
\]

(3)终止
\[
	P(O \vert \lambda)=\sum_{i=1}^N\alpha_T(i)
\]

前向算法中,步骤(1)初始化前向概率,是初始时刻的状态$i_1=q_i$和观测$o_1$的联合概率。
步骤(2)是前向概率的递推公式,
计算到时刻$t+1$部分观测序列为$o_1,o_2,\dots,o_t,o_{t+1}$
且在时刻$t+1$处于状态$q_i$的前向概率。
步骤(3)给出$P(O \vert \lambda)$的计算公式,因为
\[
	\alpha_T(i)=P(o_1,o_2,\dots,o_t,i_T=q_i \vert \lambda)
\]
所以
\[
	P(O \vert \lambda)=\sum_{i=1}^N \alpha_T(i)
\]

\section*{实验三.}

给定隐马尔科夫模型$\lambda$,定义在时刻$t$状态为$q_i$的条件下,
从$t+1$到$T$的部分观测序列为$o_{t+1},o_{t+2},\dots,o_T$的概率为后向概率,记作
\[
	\beta_t(i)=P(o_{t+1},o_{t+2},\dots,o_T \vert i_t=q_i,\lambda)
\]

可以用递推的方法求得后向概率$\beta_t(i)$及观测序列概率$P(O \vert \lambda)$。

\subsection*{观测序列概率的后向算法}

输入:隐马尔科夫模型$\lambda$,观测序列$O$

输出:观测序列概率$P(O \vert \lambda)$

(1)初值
\[
	\beta_T(i)=1,\qquad i=1,2,\dots,N
\]

(2)递推。对$t=T-1,T-2,\dots,1$
\[
	\beta_t(i)=\sum_{j=1}^N a_{ij}b_j(o_{t+1})\beta_{t+1}(j),
	\qquad i=1,2,\dots,N
\]

(3)终止
\[
	P(O \vert \lambda)=\sum_{i=1}^N \pi_ib_i(o_1)\beta_1(i)
\]

步骤(1)初始化后向概率,对最终时刻的所有状态$q_i$规定$\beta_T(i)=1$。
步骤(2)是后巷概率的递推公式。
为了计算在时刻$t$状态为$q_i$条件下时刻$t+1$之后的观测序列为
$o_{t+1},o_{t+2},\dots,o_t$的后向概率$\beta_t(i)$,
只需考虑在时刻$t+1$所有可能的$N$个状态$q_j$的转移概率,
以及在此状态下的观测$o_{t+1}$的观测概率,
然后考虑状态$q_j$之后的观测序列的后向概率。
步骤(3)求$P(O \vert \lambda)$的思路与步骤(2)一致,
只是初始概率$\pi_i$代替转移概率。

\nocite{*}
\bibliography{report}

\end{document}
