\documentclass[a4paper,UTF8]{article}
\usepackage{ctex}
\usepackage[margin=1.25in]{geometry}
\usepackage{color}
\usepackage{graphicx}
\usepackage{amssymb}
\usepackage{amsmath}
\usepackage{amsthm}
%\usepackage[thmmarks, amsmath, thref]{ntheorem}
\theoremstyle{definition}
\newtheorem*{solution}{Solution}
\newtheorem*{prove}{Proof}
\usepackage{multirow}
\usepackage{url}
\usepackage{enumerate}
\renewcommand\refname{参考文献}

%--

%--
\begin{document}
\title{实验1. 度量学习实验报告}
\author{MG1733098,周华平,\url{zhp@smail.nju.edu.cn}}
\maketitle

\section*{综述}

\section*{任务1}

\subsection*{度量函数学习目标}

根据马氏距离的定义
\[
	dist_{mah}^2(x, y) = (x - y)^\top Q(x - y) = (Ax - Ay)^\top (Ax - Ay)
\]
其中$Q$称为“度量矩阵”,而度量学习则是对$Q$的学习。
为了保持距离非负且对称,$Q$必须是(半)正定对称矩阵,即必有正交基$A$使得$Q$能写为$Q = AA^\top$。

为了提高近邻分类器的性能,我们将$Q$直接嵌入到近邻分类器的评价指标中去,
通过优化该性能目标相应地求得$Q$。
在本实验中我们采用近邻成分分析(Neighbourhood Component Analusis)进行学习。

近邻分类器在进行判别时通常使用多数投票法,领域中的每个样本投1票,
领域外的样本投0票。NCA将其替换为概率投票法,对于任意样本$x_{j}$,它对$x_{i}$分类结果影响的概率为
\[
	p_{ij} = \frac{\exp(\lVert Ax_{i} - Ax_{j} \rVert^2)}
	{\sum_{k \neq i} \exp(\lVert Ax_{i} - Ax_{k} \rVert^2)}, \qquad
	p_{ii} = 0
\]
若以留一法正确率的最大化为目标,则可计算$x_{i}$的留一法正确率,
即它被自身之外的所有样本正确分类的概率为
\[
	p_{i} = \sum_{j \in C_{i}} p_{ij}
\]
其中$C_{i} = \lbrace j \vert c_{i} = c_{j} \rbrace$,
即与$x_{i}$属于相同类别的样本的下标集合。
于是,整个样本集上的留一法正确率为
\[
	f(A) = \sum_{i} \sum_{j \in C_{i}} p_{ij} = \sum_{i} p_{i}
\]

NCA的优化目标是使得$f(A)$最大化,即
\[
	\max_{A} \sum_{i} \sum_{j \in C_{i}} p_{ij}
\]
通过求$f$对$A$的偏导,可以得到梯度公式(令$x_{ij} = x_{i} - x_{j}$)
\[
	\frac{\partial f}{\partial A} =
	-2A \sum_{i} \sum_{j \in C_{i}} p_{ij}( x_{ij} x_{ij}^\top - \sum_{k} p_{ik} x_{ik} x_{ik}^\top)
\]
根据该公式,使用梯度下降法即可求解NCA的目标函数,
得到最大化近邻分类器留一法正确率的距离度量矩阵$Q$。

\subsection*{优化算法}



\section*{任务2}
\end{document}
