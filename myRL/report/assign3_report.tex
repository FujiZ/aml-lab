\documentclass[a4paper,UTF8]{article}
\usepackage{ctex}
\usepackage[margin=1.25in]{geometry}
\usepackage{color}
\usepackage{graphicx}
\usepackage{amssymb}
\usepackage{amsmath}
\usepackage{amsthm}
%\usepackage[thmmarks, amsmath, thref]{ntheorem}
\theoremstyle{definition}
\newtheorem*{solution}{Solution}
\newtheorem*{prove}{Proof}
\usepackage{multirow}
\usepackage{url}
\usepackage{enumerate}
\usepackage{algorithm}
\usepackage{algorithmic}
\usepackage{listings}
\renewcommand{\algorithmicrequire}{\textbf{Input:}}
\renewcommand{\algorithmicensure}{\textbf{Procedure:}}
\renewcommand\refname{参考文献}

%--

%--
\begin{document}
\title{实验3. 强化学习实践}
\author{MG1733098, 周华平, \url{zhp@smail.nju.edu.cn}}
\maketitle

\section*{综述}



\section*{实验二. }



\section*{实验三. }



\subsection*{Deep Q-network(DQN)实现}

在本实验中,我使用PyTorch来实现DQN。



% TODO use pytorch
% network structure
% 贴代码?

\subsection*{CartPole}

% 参数用表格展示
% 对reward的定义

\subsection*{MountainCar}

\subsection*{Acrobot}

\section*{实验四. }



\end{document}
